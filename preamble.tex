\usepackage[utf8]{inputenc}
\usepackage{textcomp}

% -- table --
\usepackage{multirow}
\usepackage{makecell}

% -- support vietnamese --
\usepackage[vietnamese]{babel}
\addto\captionsvietnamese{\renewcommand{\figurename}{Figure}}
\addto\captionsvietnamese{\renewcommand{\tablename}{Table}}
% -- support english --
% \usepackage[english]{babel}

\usepackage[en-US]{datetime2}
\newcommand{\englishdate}{\DTMsetdatestyle{en-US}}

\usepackage{float}
\usepackage{graphicx}
\usepackage{booktabs}
% \usepackage[shortlabels]{enumitem}
\usepackage{emptypage}
\usepackage{subcaption}
\usepackage{multicol}

\usepackage{lipsum}
\usepackage{framed}
\usepackage{setspace}
\usepackage{extarrows}
\usepackage{scrextend}
\usepackage{commath}
\usepackage{algorithm}
% \usepackage[usenames, dvipsnames]{xcolor}

\usepackage{listings}
\usepackage{color}

\definecolor{dkgreen}{rgb}{0,0.6,0}
\definecolor{gray}{rgb}{0.5,0.5,0.5}
\definecolor{mauve}{rgb}{0.58,0,0.82}

% \lstset{frame=tb,
%   language=Python,
%   aboveskip=3mm,
%   belowskip=3mm,
%   showstringspaces=false,
%   columns=flexible,
%   basicstyle={\small\ttfamily},
%   numbers=none,
%   numberstyle=\tiny\color{gray},
%   keywordstyle=\color{blue},
%   commentstyle=\color{dkgreen},
%   stringstyle=\color{mauve},
%   breaklines=true,
%   breakatwhitespace=true,
%   tabsize=3
% }
% \usepackage[minted]{tcolorbox}

\usepackage{animate} % insert gif
\usepackage{caption}
\usepackage{booktabs}
\usepackage{lipsum} % To generate test text
\usepackage{ulem} % underscore, squiggly

% -- math set up --
\usepackage{amsmath, amsfonts, mathtools, amsthm, amssymb}
\usepackage{mathrsfs}
\usepackage{cancel}
\usepackage{bbm}
\usepackage{bm}
\newcommand\N{\ensuremath{\mathbb{N}}}
\newcommand\R{\ensuremath{\mathbb{R}}}
\newcommand\Z{\ensuremath{\mathbb{Z}}}
\renewcommand\O{\ensuremath{\emptyset}}
\newcommand\Q{\ensuremath{\mathbb{Q}}}
\newcommand\C{\ensuremath{\mathbb{C}}}
\newcommand\E{\ensuremath{\mathbb{E}}}
\DeclareMathOperator{\sgn}{sgn}
\usepackage{systeme}
\let\svlim\lim\def\lim{\svlim\limits}
\let\implies\Rightarrow
\let\impliedby\Leftarrow
\let\iff\Leftrightarrow
\let\epsilon\varepsilon
\usepackage{stmaryrd} % for \lightning
\newcommand\contra{\scalebox{1.1}{$\lightning$}}

% ----------------------------------------------
% tikx
\usepackage{framed}
\usepackage{tikz}
\usepackage{pgf}
\usetikzlibrary{calc,trees,positioning,arrows,chains,shapes.geometric,%
		decorations.pathreplacing,decorations.pathmorphing,shapes,%
		matrix,shapes.symbols}
\pgfmathsetseed{1} % To have predictable results

% Define a background layer, in which the parchment shape is drawn
\pgfdeclarelayer{background}
\pgfsetlayers{background, main}

% define styles for the normal border and the torn border
\tikzset{
	normal border/.style={orange!30!black!10, decorate, decoration={random steps, segment length=2.5cm, amplitude=.7mm}},
	torn border/.style={orange!30!black!5, decorate, decoration={random steps, segment length=.5cm, amplitude=1.7mm}}
    }

% Macro to draw the shape behind the text, when it fits completly in the
% page
\def\parchmentframe#1{
    \tikz{
        \node[inner sep=2em] (A) {#1};  % Draw the text of the node
        \begin{pgfonlayer}{background}  % Draw the shape behind
            \fill[normal border]
            (A.south east) -- (A.south west) --
            (A.north west) -- (A.north east) -- cycle;
\end{pgfonlayer}}}

% Macro to draw the shape, when the text will continue in next page
\def\parchmentframetop#1{
    \tikz{
        \node[inner sep=2em] (A) {#1};    % Draw the text of the node
        \begin{pgfonlayer}{background}
            \fill[normal border]              % Draw the ``complete shape'' behind
            (A.south east) -- (A.south west) --
            (A.north west) -- (A.north east) -- cycle;
            \fill[torn border]                % Add the torn lower border
            ($(A.south east)-(0,.2)$) -- ($(A.south west)-(0,.2)$) --
            ($(A.south west)+(0,.2)$) -- ($(A.south east)+(0,.2)$) -- cycle;
\end{pgfonlayer}}}

% Macro to draw the shape, when the text continues from previous page
\def\parchmentframebottom#1{
    \tikz{
        \node[inner sep=2em] (A) {#1};   % Draw the text of the node
        \begin{pgfonlayer}{background}
            \fill[normal border]             % Draw the ``complete shape'' behind
            (A.south east) -- (A.south west) --
            (A.north west) -- (A.north east) -- cycle;
            \fill[torn border]               % Add the torn upper border
            ($(A.north east)-(0,.2)$) -- ($(A.north west)-(0,.2)$) --
            ($(A.north west)+(0,.2)$) -- ($(A.north east)+(0,.2)$) -- cycle;
\end{pgfonlayer}}}

% Macro to draw the shape, when both the text continues from previous page
% and it will continue in next page
\def\parchmentframemiddle#1{
    \tikz{
        \node[inner sep=2em] (A) {#1};   % Draw the text of the node
        \begin{pgfonlayer}{background}
            \fill[normal border]             % Draw the ``complete shape'' behind
            (A.south east) -- (A.south west) --
            (A.north west) -- (A.north east) -- cycle;
            \fill[torn border]               % Add the torn lower border
            ($(A.south east)-(0,.2)$) -- ($(A.south west)-(0,.2)$) --
            ($(A.south west)+(0,.2)$) -- ($(A.south east)+(0,.2)$) -- cycle;
            \fill[torn border]               % Add the torn upper border
            ($(A.north east)-(0,.2)$) -- ($(A.north west)-(0,.2)$) --
            ($(A.north west)+(0,.2)$) -- ($(A.north east)+(0,.2)$) -- cycle;
\end{pgfonlayer}}}


% Define the environment which puts the frame
% In this case, the environment also accepts an argument with an optional
% title (which defaults to ``Example'', which is typeset in a box overlaid
% on the top border
\newenvironment{parchment}[1][Example]{%
    \def\FrameCommand{\parchmentframe}%
    \def\FirstFrameCommand{\parchmentframetop}%
    \def\LastFrameCommand{\parchmentframebottom}%
    \def\MidFrameCommand{\parchmentframemiddle}%
    \vskip\baselineskip
    \MakeFramed {\FrameRestore}
    \noindent\tikz\node[inner sep=1ex, draw=black!20,fill=white,
    anchor=west, overlay] at (0em, 2em) {\sffamily#1};\par}%
{\endMakeFramed}

% ----------------------------------------------
\mode<presentation>{
    % \usetheme{Warsaw}
    % Boadilla CambridgeUS
    % default Antibes Berlin Copenhagen
    % Madrid Montpelier Ilmenau Malmoe
    % Berkeley Singapore Warsaw
    \usecolortheme{whale}
    % beetle, beaver, orchid, whale, dolphin
    % \useoutertheme{infolines}
    % infolines miniframes shadow sidebar smoothbars smoothtree split tree
    \useinnertheme{circles}
    % circles, rectanges, rounded, inmargin
}
% Can define your own color with HTML code
\definecolor{framecolorright}{HTML}{014c8d}
\definecolor{structurecolor}{HTML}{0065BD}
\colorlet{doge_main}{structurecolor}
\colorlet{sharky_secondary}{white}
\definecolor{sharky_black}{RGB}{43,40,40}

% Define color
\setbeamercolor{title page top}{fg=sharky_secondary, bg=doge_main}
\setbeamercolor{title page bottom}{fg=sharky_black, bg=sharky_secondary}
\setbeamercolor{author in head/foot}{fg=sharky_secondary, bg=doge_main}    
\setbeamercolor{date in head/foot}{fg=sharky_secondary, bg=doge_main}  
% \setbeamercolor{section in head/foot}{fg=sharky_secondary, bg=sharky_main}
% \setbeamercolor{title page bottom}{fg=sharky_black, bg=sharky_secondary}
\setbeamercolor{frametitle right}{fg=sharky_secondary, bg=doge_main}
\setbeamercolor{structure}{fg=structurecolor}

% set block color
% \setbeamercolor{block title}{bg=blue!80,fg=white}

% Set beamer font
% \setbeamerfont{frametitle}{size=\Large, series=\bfseries}
\setbeamerfont{title}{series=\bfseries}
% \setbeamerfont{subtitle}{size=\small}
% \setbeamerfont{author}{}
% \setbeamerfont{date}{}

\newcommand{\reditem}[1]{\setbeamercolor{item}{fg=red}\item #1}

% scale formula size
\newcommand*{\Scale}[2][4]{\scalebox{#1}{\ensuremath{#2}}}

% font warning
\renewcommand\textbullet{\ensuremath{\bullet}}

\setbeamertemplate{headline}
{
    \leavevmode%
    \hbox{%
    \begin{beamercolorbox}[wd=.5\paperwidth,ht=2.25ex,dp=1ex,right,rightskip=1em]{section in head/foot}%
        \usebeamerfont{subsection in head/foot}\hspace*{2ex}\insertsectionhead
    \end{beamercolorbox}%
    \begin{beamercolorbox}[wd=.5\paperwidth,ht=2.25ex,dp=1ex,left,leftskip=1em]{subsection in head/foot}%
        \usebeamerfont{section in head/foot}\insertsubsectionhead\hspace*{2ex}
    \end{beamercolorbox}}%
    \vskip0pt%
}
% ---------------------------------------------------------------------

\AtBeginSection[]
{
    \setbeamertemplate{navigation symbols}{}
    \frame[plain,c,noframenumbering]{
        \sectionpage
        \tableofcontents[currentsection,subsectionstyle=hide]}
    \setbeamertemplate{navigation symbols}{\normalsize}
}

% ---------------------------------------------------------------------
% flow chart
\tikzset{
    >=stealth',
    punktchain/.style={
        rectangle,
        rounded corners,
        % fill=black!10,
        draw=white, very thick,
        text width=6em,
        minimum height=2em,
        text centered,
        on chain
    },
    largepunktchain/.style={
        rectangle,
        rounded corners,
        draw=white, very thick,
        text width=10em,
        minimum height=2em,
        on chain
    },
    line/.style={draw, thick, <-},
    element/.style={
        tape,
        top color=white,
        bottom color=blue!50!black!60!,
        minimum width=6em,
        draw=blue!40!black!90, very thick,
        text width=6em,
        minimum height=2em,
        text centered,
        on chain
    },
    every join/.style={->, thick,shorten >=1pt},
    decoration={brace},
    tuborg/.style={decorate},
    tubnode/.style={midway, right=2pt},
    font={\fontsize{10pt}{12}\selectfont},
}

% ---------------------------------------------------------------------
% code setting
\lstset{
    language=C++,
    basicstyle=\ttfamily\footnotesize,
    keywordstyle=\color{red},
    breaklines=true,
    xleftmargin=2em,
    numbers=left,
    numberstyle=\color[RGB]{222,155,81},
    frame=leftline,
    tabsize=4,
    breakatwhitespace=false,
    showspaces=false,
    showstringspaces=false,
    showtabs=false,
    morekeywords={Str, Num, List},
}

\makeatletter

\newdimen\sharky@frametitlesep
\sharky@frametitlesep=1.ex
\newdimen\sharky@footbarsep
\sharky@footbarsep=1ex


% disable navigation
\setbeamertemplate{navigation symbols}{}

% custom draw the title page above
\date{August 3, 2025}
\setbeamertemplate{title page}{%
    \vskip-0.5cm%
    \begin{beamercolorbox}[wd=\paperwidth, ht=0.5\paperheight, center, sep=0pt]{title page top}
        \usebeamerfont{title}\inserttitle\par
        \vspace{2em}%
        \usebeamerfont{subtitle}\insertsubtitle\par
        \vspace{1em}%
    \end{beamercolorbox}%
    \begin{beamercolorbox}[wd=\paperwidth, ht=0.5\paperheight, center]{title page bottom}
        \vbox to 0.5\paperheight{%
        \vfil%
        \usebeamerfont{author}\insertauthor\par%
        \vspace{0.5em}%
        {\insertinstitute}\par
        \vspace{3em}%
        \usebeamerfont{date}\insertdate\par%
        \vfil%
    }%
    \end{beamercolorbox}
}

\setbeamertemplate{frametitle}{%
\nointerlineskip%
\usebeamerfont{frametitle}%
    \begin{beamercolorbox}[wd=\paperwidth,sep=\sharky@frametitlesep]{frametitle}
        \usebeamerfont{frametitle}\insertframetitle%
    \end{beamercolorbox}
}

%
\setbeamertemplate{caption}[numbered]

% footer
\makeatletter
\setbeamertemplate{footline}
{
    \leavevmode%
    \hbox{%
    \begin{beamercolorbox}[wd=.5\paperwidth,ht=2.25ex,dp=1ex,center]{author in head/foot}%
        \usebeamerfont{author in head/foot}~~\beamer@ifempty{}{}{(\insertshortinstitute)}
    \end{beamercolorbox}%
    \begin{beamercolorbox}[wd=.5\paperwidth,ht=2.25ex,dp=1ex,right]{date in head/foot}%
        \insertframenumber{} / \inserttotalframenumber\hspace*{2ex} 
    \end{beamercolorbox}}%
    \vskip0pt%
}
\makeatother

% -- Theorem set up --
\makeatother
\usepackage{thmtools}
\usepackage[framemethod=TikZ]{mdframed}
\mdfsetup{skipabove=1em,skipbelow=0em, innertopmargin=5pt, innerbottommargin=6pt}

\theoremstyle{definition}

\makeatletter

\@ifclasswith{report}{nocolor}{
    \declaretheoremstyle[headfont=\bfseries\sffamily, bodyfont=\normalfont, mdframed={ nobreak } ]{thmgreenbox}
    \declaretheoremstyle[headfont=\bfseries\sffamily, bodyfont=\normalfont, mdframed={ nobreak } ]{thmredbox}
    \declaretheoremstyle[headfont=\bfseries\sffamily, bodyfont=\normalfont]{thmbluebox}
    \declaretheoremstyle[headfont=\bfseries\sffamily, bodyfont=\normalfont]{thmblueline}
    \declaretheoremstyle[headfont=\bfseries\sffamily, bodyfont=\normalfont, numbered=no, mdframed={ rightline=false, topline=false, bottomline=false, }, qed=\qedsymbol ]{thmproofbox}
    \declaretheoremstyle[headfont=\bfseries\sffamily, bodyfont=\normalfont, numbered=no, mdframed={ nobreak, rightline=false, topline=false, bottomline=false } ]{thmexplanationbox}
    \AtEndEnvironment{eg}{\null\hfill$\diamond$}%
}{
    \declaretheoremstyle[
        headfont=\bfseries\sffamily\color{doge_main!70!black}, bodyfont=\normalfont,
        mdframed={
            linewidth=2pt,
            rightline=false, topline=false, bottomline=false,
            linecolor=ForestGreen, backgroundcolor=ForestGreen!10,
        }
    ]{thmdogebox}
    
    \declaretheoremstyle[
        headfont=\bfseries\sffamily\color{ForestGreen!70!black}, bodyfont=\normalfont,
        mdframed={
            linewidth=2pt,
            rightline=false, topline=false, bottomline=false,
            linecolor=ForestGreen, backgroundcolor=ForestGreen!10,
        }
    ]{thmgreenbox}

    \declaretheoremstyle[
        headfont=\bfseries\sffamily\color{NavyBlue!70!black}, bodyfont=\normalfont,
        mdframed={
            linewidth=2pt,
            rightline=false, topline=false, bottomline=false,
            linecolor=NavyBlue, backgroundcolor=NavyBlue!5,
        }
    ]{thmbluebox}

    \declaretheoremstyle[
        headfont=\bfseries\sffamily\color{NavyBlue!70!black}, bodyfont=\normalfont,
        mdframed={
            linewidth=2pt,
            rightline=false, topline=false, bottomline=false,
            linecolor=NavyBlue
        }
    ]{thmblueline}

    \declaretheoremstyle[
        headfont=\bfseries\sffamily\color{RawSienna!70!black}, bodyfont=\normalfont,
        mdframed={
            linewidth=2pt,
            rightline=false, topline=false, bottomline=false,
            linecolor=RawSienna, backgroundcolor=RawSienna!10,
        }
    ]{thmredbox}

    \declaretheoremstyle[
        headfont=\bfseries\sffamily\color{RawSienna!70!black}, bodyfont=\normalfont,
        numbered=no,
        mdframed={
            linewidth=2pt,
            rightline=false, topline=false, bottomline=false,
            linecolor=RawSienna, backgroundcolor=RawSienna!1,
        },
        qed=\qedsymbol
    ]{thmproofbox}

    \declaretheoremstyle[
        headfont=\bfseries\sffamily\color{NavyBlue!70!black}, bodyfont=\normalfont,
        numbered=no,
        mdframed={
            linewidth=2pt,
            rightline=false, topline=false, bottomline=false,
            linecolor=NavyBlue, backgroundcolor=NavyBlue!1,
        },
    ]{thmexplanationbox}
}

% -- block stuff --

\setbeamercolor{block title alerted}{fg = white,
                                     bg = dkgreen}
\setbeamercolor{block body alerted} {bg = blockbody}
\definecolor{blockbody} {cmyk}{0, 0, 0, 0.2}

% -- use for english --

\declaretheorem[style=thmgreenbox, name=Definition]{definition}
\declaretheorem[style=thmbluebox, numbered=no, name=Example]{eg}
\declaretheorem[style=thmbluebox, numbered=no, name=Exercise]{ex}
\declaretheorem[style=thmredbox, name=Proposition]{prop}
\declaretheorem[style=thmredbox, name=Theorem]{theorem}
\declaretheorem[style=thmredbox, name=Lemma]{lemma}
\declaretheorem[style=thmredbox, numbered=no, name=Corollary]{corollary}
\declaretheorem[style=hmdogebox, name=Proof]{prof}

\@ifclasswith{report}{nocolor}{
    \declaretheorem[style=thmproofbox, name=Proof]{replacementproof}
    \declaretheorem[style=thmexplanationbox, name=Proof]{explanation}
    \renewenvironment{proof}[1][\proofname]{\begin{replacementproof}}{\end{replacementproof}}
}{
    \declaretheorem[style=thmproofbox, name=Proof]{replacementproof}
    \renewenvironment{proof}[1][\proofname]{\vspace{-10pt}\begin{replacementproof}}{\end{replacementproof}}

    \declaretheorem[style=thmexplanationbox, name=Proof]{tmpexplanation}
    \newenvironment{explanation}[1][]{\vspace{-10pt}\begin{tmpexplanation}}{\end{tmpexplanation}}
}

% -- use for vietnamese --

\declaretheorem[style=thmgreenbox, name=Định nghĩa]{defivn}
\declaretheorem[style=thmbluebox, numbered=no, name=Ví dụ]{egvn}
\declaretheorem[style=thmbluebox, numbered=no, name=Bài tập]{exvn}
\declaretheorem[style=thmredbox, name=Mệnh đề]{propvn}
\declaretheorem[style=thmredbox, name=Định lý]{theovn}
\declaretheorem[style=thmredbox, name=Bổ đề]{lemmavn}
\declaretheorem[style=thmredbox, numbered=no, name=Hệ quả]{corollaryvn}
\declaretheorem[style=thmblueline, numbered=no, name=Nhận xét]{remarkvn}

\declaretheorem[style=thmexplanationbox, name=Giải]{replacementsolvevn}
\newenvironment{solvevn}[1][\proofname]{\vspace{-10pt}\begin{replacementsolvevn}}{\end{replacementsolvevn}}

\@ifclasswith{report}{nocolor}{
    \declaretheorem[style=thmproofbox, name=Chứng minh]{replacementproofvn}
    \newenvironment{proofvn}[1][\proofname]{\begin{replacementproofvn}}{\end{replacementproofvn}}
}{
    \declaretheorem[style=thmproofbox, name=Chứng minh]{replacementproofvn}
    \newenvironment{proofvn}[1][\proofname]{\vspace{-10pt}\begin{replacementproofvn}}{\end{replacementproofvn}}
}

\newtheorem*{probvn}{Bài toán}


% ------------------------

\makeatother

\declaretheorem[style=thmblueline, numbered=no, name=Remark]{remark}
% \declaretheorem[style=thmblueline, numbered=no, name=Note]{note}

\newtheorem*{uovt}{UOVT}
\newtheorem*{notation}{Notation}
\newtheorem*{previouslyseen}{As previously seen}
\newtheorem*{problem}{Problem}
\newtheorem*{observe}{Observe}
\newtheorem*{property}{Property}
\newtheorem*{intuition}{Intuition}

\usepackage{tabularray}
\newcommand{\rotvert}{\rotatebox[origin=c]{90}{$\vert$}}
\newcommand{\rowsvdots}{\multicolumn{1}{@{}c@{}}{\vdots}}
\newcommand{\brows}[1]{%
    \begin{bmatrix}
    \begin{array}{@{\protect\rotvert\;}c@{\;\protect\rotvert}}
    #1
    \end{array}
    \end{bmatrix}
}
